\documentclass[a4paper,12pt]{article}
\usepackage[utf8]{inputenc}
\usepackage[T1]{fontenc}
\usepackage[french]{babel}
\usepackage{graphicx}
\usepackage{url}

% ajout des librairie pour l'insertion de code
\usepackage{xcolor}
\usepackage{listings}
\lstset{
	basicstyle=\ttfamily,
	stringstyle=\ttfamily\color{green!50!black},
	keywordstyle=\color{blue}\bfseries,
	commentstyle=\color{red!50!black}\itshape,
	showstringspaces=true,
	tabsize=2, frame=single,
	numbers=left, numberstyle=\tiny,
	firstnumber=1, stepnumber=1, numbersep=5pt
	}

% En-tête et pied de page
\usepackage{fancyhdr}
\pagestyle{fancy}

% Déclaration de l'en-tête
\renewcommand{\headrulewidth}{1pt}
\fancyhead[l]{Cube led}
\fancyhead[c]{Documentation}
\fancyhead[r]{\today}

% Déclaration du pied de page
\renewcommand{\footrulewidth}{1pt}
\fancyfoot[l]{Kevin Amando \& \\ Alan Devaud \& \\ Gregory Mendez}
\fancyfoot[c]{T.IS-E2B}
\fancyfoot[r]{page \thepage}

% Information sur le projet (auteur, titre, date,etc.)
\title{Cube à led}
\author{Kevin Amando \& Alan Devaud \& Gregory Mendez}
\date{\today}

\begin{document}

% Title page
\begin{titlepage}
    \begin{center}
        % Logo and some informations
        {\large CFPT Ecole d'informatique - Technicien ES en informatique}\\[0.5cm]
        {\large Travail de semestre inter-degré 2016-2017}\\[0.5cm]
        %\includegraphics[width=0.6\textwidth]{YGCLogo.png}\\[1cm]
        
        % Title
        \rule{\linewidth}{0.5mm} \\[0.4cm]
        { \huge \bfseries Cube à Led \\ Documentation \\[0.4cm] }
        \rule{\linewidth}{0.5mm} \\[1.5cm]
    
        % Author and supervisor
        \noindent
        \begin{minipage}{0.4\textwidth}
          \begin{flushleft} \large
            \emph{Elèves :}\\
            M. Kevin \textsc{Amado} \\
            M. Alan \textsc{Devaud}\\
            M. Gregory \textsc{Mendez}
          \end{flushleft}
        \end{minipage}%
        \begin{minipage}{0.4\textwidth}
          \begin{flushright} \large
            \emph{Enseignants :} \\
            M.~Denis \textsc{Carbone}\\
            M.~Nicolas \textsc{Wanner}
          \end{flushright}
        \end{minipage}
        
        \vfill

        % Bottom of the page
        {\large Version 1.0 du\\ \today}
    \end{center}
\end{titlepage}

\newpage

%Résumé
\section{Résumé}
\newpage

% Table of contents
\tableofcontents
\newpage

% Introduction
\section{Introduction}
Notre projet consiste à améliorer le projet Cube de M. \textsc{Aubert} Jonathan dans le câdre de l'atelier Technicien, qui regroupe les deuxièmes année avec les premières. Nous sommes trois à travailler ensemble. L'objectif est de reprendre le travail réaliser en ajoutant une vue 3D. Notre application doit pouvoir gérer la couleur des Leds. La gestion d'animation est aussi demandée.
\newpage

% Cahier des charges
\section{Cahier des charges}
\newpage

% Analyse de l'existant
\section{Analyse de l'existant}
Le Cube à Leds est représenté sous la forme d'un tableau à trois dimensions :
\begin{center}
	\textsc{CubeLED}[x][y][]nbImage = value
\end{center}
\begin{tabular}{r  l}
	\raggedright{\textbf{CubeLED :}} & Le nom du tableau\\
	\textbf{x :} & Position d'une Led sur l'axe "x"\\
	\textbf{y :} & Position d'une Led sur l'axe "y"\\
	\textbf{nbImage :} & Numéro de l'image (animation)\\
	\textbf{value :} & Valeur stockée\\
\end{tabular}

\newpage

% Analyse fonctionnelle
\section{Analyse fonctionnelle}
\newpage

% Analyse organique
\section{Analyse organique}
\newpage

% Conclusion
\section{Conclusion}

\newpage

% Pictures table
\newpage
\listoffigures


\end{document}